\documentclass{article}
\usepackage[utf8]{inputenc}

\title{LAB2 Report}
\author{Chibu G. Moganedi, Sammy Maakwana, Ntokozo Gule, Thabang Khoza  }
\date{September 2018}

\begin{document}

\maketitle

\section{Introduction}
\text{The TestProg class is the one with the main function.
First it checks if the command line argument is passed.
\vspace{5mm}


This command line argument needs to be the path to the file containing the test
X value cases, this X values need to be integers (or long) written line after
line in the passed file.
This is done to stream-line the process of testing multiple cases.
\vspace{5mm}

Then the ListOfPrimes class is the one that performs the sieve of Eratosthenes
algorithm to fine the prime numbers up to the given X value of each line in
the input file passed on the command line.
\vspace{5mm}

Now for each of the test cases (X) a list of the prime numbers are written to
a text file with then name X. The prime numbers are stored in this files for
each line of the input. These files are written into the the output directory.
}

\section{Modifications to the way parameters are passed}
\text{We decided to pass our parameters as a file with a list of values that the user wants to test with. So if the user would like to test one or more values then the user would enter those values in a text file and when calling the TestProg, the user will add the text file path as an argument.
\vspace{5mm}
}

\section{Predicted test cases}
\text{Tested cases were for values: 10, 100, 1000, 10 000, 100 000, 1 000 000 and 10 000 000.
For results, please go to the link \vspace{5mm}

https://github.com/cgmoganedi/Eratosthenes/tree/master/src/outputs.
\vspace{5mm}
}
\section{Input and Output files}
\text{Input and Output files are at the link \vspace{5mm}

https://github.com/cgmoganedi/Eratosthenes/tree/master/src/.
\vspace{5mm}
}

\section{}

\end{document}
